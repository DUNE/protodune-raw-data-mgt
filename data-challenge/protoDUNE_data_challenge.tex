\documentclass[pdftex,12pt,letter]{article}
\usepackage[binary-units=true]{siunitx}
\usepackage[margin=0.75in]{geometry}
\usepackage{verbatim}
\usepackage{graphicx}
\usepackage{cite}
\usepackage{color}
\usepackage[pdftex,pdfpagelabels,bookmarks,hyperindex,hyperfigures]{hyperref}
\usepackage{xspace}
\usepackage{amssymb}


\bibliographystyle{unsrt}

\newcommand{\fixme}[1]{\textbf{FIXME: #1}}    
\newcommand{\pd}{protoDUNE\xspace}

\title{A Proposal for the Single-Phase \pd Experiment Data Challenges in 2017-2018}
\date{\today}
\author{R.Pordes, M.Potekhin, R.Sulej}

\begin{document}


\maketitle

\begin{abstract}



\noindent It is important to ensure 100\% readiness of the
end-to-end \pd software and computing complex during the detector commissioning period and
throughout data taking in 2018. We propose to conduct two Data Challenges (DCs) in order to identify
and address potential issues before they can impact the experiment.
DCs will include defining and  testing interfaces, measuring rates and reliability of data transfers
and job execution, understanding the set of processing workflows, and publishing the results of monitoring
the infrastructure as well as the physics outputs.


\end{abstract}

\tableofcontents

\pagebreak

\section{About this document}
This document is work in progress. It was first created in May 2017 descibing the initial components
of the data challenges and it is being continuously updated to reflect work that is being done in this area.


The goal of this document is to provide concise information about the proposed \pd Data Challenges
to the individual working groups in order to coordinate effort and come to a consensus as to the scope,
plans and schedule of the proposed Data Challenges. \textit{It does not contain detailed descriptions
and/or designs of the \pd computing
infrastructure elements} and the reader is reffered to existing documentation where
needed, with references provided in the text.
General information regarding the \pd (NP04) computing is summarized in the \pd-SP Technical Design Report\cite{docdb1794}.

\section{The Scope of the Data Challenges}
\subsection{Assumptions about the software status and provisioning}
The Data Challenges will be done with the understanding that the software is not in its final state/readiness. It is assumed
however that its platform is finalized as mainly based on LArSoft and the mechanism of software provisioning which leverage
CVMFS are in place and close to final.

\subsection{Components}
The following components (including infrastructure, middleware and software) can be identified as relevant in the context
of Data Challenges (which will be also referred to as \textit{``DC''} in this document):
\begin{enumerate}
\item The DAQ \textbf{Online Buffer} which stores raw data assembled by the Event Builders, as files on disk. The internals
of the DAQ system itself are not within the scope.

\item  \textbf{Beam Instrumentation Interface}. Most of the Beam Instrumentation data is not included into the data stream
handled by the \pd DAQ (whose primary task is to capture data from the TPC and the Photon Detector), and is instead
transmitted to the Beam Instrumentation Database at CERN. It is not available immediately for each triggered event,
and is instead available in batches after each spill cycle of the SPS. Interfacing this system is a task that needs
to be addressed for \pd to be able to include these data in its offline processing in an optimal manner.
%validate its beam trigger system.

% Adding Online Monitoring based on talking to Christos/Flavio and as a test for writing to the git repository
% Ruth, July 31st 2017
\item The \textbf{Slow Control and Online Monitoring System} which provide control, monitoring and display of the data acquisition readout and detector parameters. 

\item The \textbf{Data Management System}\cite{docdb1212}  which performs data transfers between a few endpoints
at CERN and FNAL starting with the Online Buffer, including output from DQM,  and which is also tasked with proper accounting and handling of the file catalog and other metadata
by interacting with the SAM system at FNAL (see next item). The Data Management/Handling System will also interface the disk-based mass storage
at CERN and FNAL (i.e.\,EOS\cite{eos} and dCache) as well as tape systems (CASTOR and Enstore respectively).
We anticipate using the \textit{Fermi File Transfer System} (also abbreviated as F-FTS\cite{fts}).

\item The \textbf{File Catalog (Metadata)}  -- SAM system at FNAL which comprises the functionality of the file catalog, data storage and
retrieval based on Metadata and can also be used for orchestration of production workflows.

\item \textbf{Data Quality Monitoring} (DQM) which is tasked with running algorithms with turnaround time short enough to provide
actionable results (under and hour) but which won't fit into the computational footprint of the Online Monitoring (a part of DAQ).
To support DQM, the ProtoDUNE Prompt Processing System (\textbf{p3s}) \cite{docdb1811,p3s} has been created and
is capable ofl running any type ofDQM workload as formulated and programmed by the working groups. A dedicated Web
service will provide access to visual and data products coming out of DQM.

\item \textbf{Calibration} While it is not expected that these components will be finalized
(and some even exist) at the time of the first Data Challenge it is important to have a firm grasp of the required interfaces,
data flow patterns and other crucial aspects of the \pd Calibration software.
 \item \textbf{Reconstruction}. While it is not expected that these components will be finalized
(and some even exist) at the time of the first Data Challenge it is important to have a firm grasp of the required interfaces,
data flow patterns and other crucial aspects of the \pd Reconstruction Systems.
%Having prototypes in place is therefore crucial on the time scale of the Data Challenges.
An optional (but desirable)  component of the overall production chain is the \textit{Data Reduction} step along the lines described
in \cite{docdb2089}.
% It combines a few signal-processing procedures (cf. digital filtering and noise reduction), and it will be
%necessary in any version or architecture of the production software.
%Understanding the interfaces and practical implications of this component would be a useful part of the Data Challenge.

\item \textbf{Analysis Suite Prototypes}. Same comment applies here as to the previous item, i.e.\,while there is no expectation
for the final design and implementation of the analysis chain to exist early in the experiment (especially that it is by nature the most
dynamic and fluid part of all software) it is important to put in place, document and optimize its interfaces with various components
of infrastructure e.g.\,calibration databases, Metadata, software and data provenance controls etc.

\item \textbf{Production Operations Management System (based on POMS)}. This FNAL-based service includes using the DUNE service portal, production workflows executed by POMS \cite{poms}, including the management and automation of  jobs
submission on distributed resources on the Grid. It will be the primary platform for \pd on which to run production and analysis.


\end{enumerate}
\noindent The relationship between the various infrastructure and other components listed above is schematically illustrated in
Fig.\ref{fig:dc1}. This diagram relflects the fact that transmission of \pd data is a multistage process and in particular there
will be separate instances of F-FTS transferring the data from the Online Buffer to mass storage at CERN, and then from
CERN to FNAL and elsewhere.

\begin{figure}[tbh]
  \centering
  \includegraphics[width=1.0\textwidth]{../figures/data_challenge_1.pdf}
  \caption{Schematics of data flow and  processing in \pd data challenges.}
  \label{fig:dc1}
\end{figure}
\noindent Data and production streams which belong to \pd Monte Carlo simulation domain are not included in the
scope of Data Challenges presented here.



\subsection{Work-up to the Data Challenges}
Before any end-to-end data challenge can be meaningful, all individual components and interfaces must undergo
their own functional testing. The scope of the pre-DC activity includes most of the components above and an
understanding of what files, file and data formats, and interfaces are going to be used. An initial list is given below and will be modified as things are better understood.
\begin{itemize}

\item \textbf{Slow Control and Online Monitoring}
\begin{itemize}
\item \emph{DC readiness: not a part of DC}
\end{itemize}

\item \textbf{Online Budder }
\begin{itemize}
\item \emph{DC readiness: won't be ready for DC1}
\end{itemize}

\item \textbf{Data Transfer and SAM interaction}.  Both F-FTS1 and F-FTS2 components as schematically depicted in Fig.\,\ref{fig:dc1}
must undergo end-to-end functional testing including registration in SAM and sinking data to tape storage both at CERN and FNAL. A test
agent will be created which simulates creation of the raw data files by DAQ, which are written to the Online Buffer and picked up
by the F-FTS for transmission.
\begin{itemize}
\item \emph{DC readiness: done}
\end{itemize}

\item \textbf{Beam Instrumentation Interface}. At a minimum, prototype schemas for \pd beam detectors must be created in
the CERN Beam Instrumentation Database, and populated with dummy data. An extraction process must be put in place which
``captures'' these data and commits it either to a database managed by \pd and/or as files on disk (EOS). A prototype
of the merging process must be put in place. This will also require data access libraries to be utilized in the production process.
\begin{itemize}
\item \emph{DC readiness: new design in the works with ``collecors'' running at CERN and data transmitted to FNAL - sole copy kept at FNAL; won't be used in DC1}
\end{itemize}

% \newpage
\item  \textbf{DQM/p3s}. The prompt processing system must be tested and validated to run with the primary data
staging area being in EOS. The issues of deployment, updating and maintenance of LArSoft and other requisite software
must be resolved.
\begin{itemize}
\item \emph{DC readiness: ready and tested at CERN}
\end{itemize}


\item  \textbf{Data Reduction Software}.
A prototype of this software module must be tested with simulated data and packaged in a way that is portable across
FNAL and facilities at CERN.
\begin{itemize}
\item \emph{DC readiness: not currently in scope}
\end{itemize}

\item  \textbf{Calibration}
Prototypes of this software need to be put in place. They will
include components specific to LArTPC, the Photon Detector and the Beam Instrumentation data, which will also require
merging and processing.

\begin{itemize}
\item \emph{DC readiness: not currently in scope but basic components will be added soon}
\end{itemize}


\item \textbf{Production Software} Prototypes of this software need to be put in place. They will
include components specific to LArTPC, the Photon Detector and the Beam Instrumentation data, which will also require
merging and processing.

\begin{itemize}
\item \emph{DC readiness: given that DC1 is MC-based and LArSoft payloads were tested extensively we assume that it is in place}
\end{itemize}

\item  \textbf{Production Management and Support}. This includes running prototypes of this software utilizing the FNAL POMS.
\begin{itemize}
\item \emph{DC readiness: POMS in place, coordination is required Attn: Anna Mazzacane}
\end{itemize}


\item  \textbf{Analysis}. The main point in this exercise to to ready the interface of the analysis software to SAM and other
components of the production system(s). There are no strict requirements to functionality and performance of the analysis
software at this stage.

\begin{itemize}
\item \emph{DC readiness: not currently in scope}
\end{itemize}

\end{itemize}
\noindent  One of the goals of the work-up period is to identify components that overlap and where software can be shared and reused. 

%\subsection{Collaboration Tools, Reporting and Documentation}
%We are using a mail list (DUNE-proto-sp-dc) to communicate.
%, DUNE SLACK channel (pd-sp-datachallenges) for planning and technical details.
 %Documentation to be retained will be submitted to the DUNE DocDB. Results/summary of the data challenge for the will be posted to the DUNE wiki.

\clearpage

\section{DC 1: a proposed scenario}

Data Challenge 1 (DC1) is taking place the week of November 6th.
It is decoupled from the Cold Box Test taking place in Fall 2017 and aims to minimize impact on same.
Initial deployment of F-FTS components will be on a CERN OpenStack VM for testing of functionality
but not necessarily of performance. Proposed steps:
\begin{itemize}

\item Provide Input Data: 
\begin{itemize}
\item Simulated raw data files will be deposited into an \emph{emulated Online Buffer ( an EOS parition)} by a specially created test agent.
\item A script will be written to deposit or cycle through several files. 
\end{itemize}
\item SAM Metadata is created for the files either by hand or through a script

\item The data  are automatically detected by F-FTS-Lite and transmitted to an EOS dropbox.

\item The input data files are transferred to a few storage locations at Fermilab and CERN and
declared to the file catalog.
\begin{itemize}
\item F-FTS initiates data transfer to dCache and Enstore at FNAL
\item F-FTS initiates data transfer to the input directory of p3s for processing, these directory acts similar to a dropbox and files are then managed by p3s
\item F-FTS initiates data transfer to CASTOR (ongoing discussions of access to tape)
\item Files are updated in SAM
\end{itemize}

\item %The original plan called for the Beam Instrumentation data to be captured
%by \pd systems and stored either on a DB server or as files on EOS.
%These data do not need to be real at this point, as the purpose of this step is to test interfaces and protocols.
The current proposal is to replicate the BI data to FNAL which would be then the only
source of offline processing including DQM. This may not be ready in time for DC1

\item Automated process initiates DQM streams in p3s which operates using
CERN Tier-0\,\cite{lxbatch}.

\item DQM results (e.g. event display and purity tracker) are provided to users via a dedicated Web service.

\item \textbf{Production team at FNAL submits production jobs using the newly arrived data as input.}
This primarily includes reading the LArTPC data, and the Photon Detector and Beam Instrumentation
systems will be included if ready in time.

\item The Neutrino Platform cluster\,\cite{neut} will not be used in DC1.

\end{itemize}

\clearpage

\section{DC2: TBD}
This will be listed nearer the date - at the moment sometime in March 2018.
%\begin{itemize}
%\item DAQ sends raw data (such as pulser or simulated trigger data) to the Online Bufer.
%\item The data is detected by F-FTS1 and transmitted to EOS.
%\item F-FTS2 initiates data transfer to dCache at FNAL. Contact is made with the Metadata system (SAM) where the files are registered.
%\item Realistic Beam Instrumentation data is captured by \pd systems and is stored either on a DB server or as files on EOS.
%\item Automated process initiates DQM streams in p3s using  CERN Tier-0. At a minimum, ADC and noise spectra (FFT) are produced and delivered to the user by a Web service.
%\item Production team at FNAL submits production jobs using the newly arrived data as input. Interfaces between software components,
data etc as well as actual functionaltiy are close to final.
%\item Analysis chain is activated.
%\end{itemize}


%\section{Personnel and Responsibilities}
%\color{red} This list is currently being revised. \color{black}
%\begin{itemize}
%\item Online Buffer and the test agent -- DAQ team
%\item Beam Instrumentation Interface -- Beam Instrumentation Group, DB coordinator: J.Paley
%\item F-FTS -- FNAL team (lead: A.Norman)
%\item p3s -- prompt processing team (lead: M.Potekhin)
%\item DQM and production payloads -- DRA team (co-leads: D.Stefan, R.Sulej)
%\item Calibrations -- M.Mooney
%\item Integration consultant -- B.Viren
%\item Metadata and SAM -- FNAL SCD (T.Junk, S.Fuess)
%\item Production support at CERN and FNAL -- TBD
%\item Analysis -- TBD
%\item Coordination -- R.Pordes
%\end{itemize}

\clearpage

\appendix
\section{Files and Interfaces}
\emph{Information here is still under development and subject to discussion. In general it will be pointers to existing documentation  internal to the appropriate components. }
\\ 

\begin{figure}[tbh]
  \centering
  \includegraphics[width=1.0\textwidth]{../figures/dc1_integration.png}
  \caption{Schematic of Dependencies}
  \label{fig:dependencies}
\end{figure}

\renewcommand{\theenumi}{\alph{enumi}}
\begin{enumerate}
\item Beam Instrumentation Database Interface - data format and name of code module and description of the database attributes and internface.
\item Calibration data  -  data format and name of code module and description of the database attributes and interface for calibration information.
\item DQM output files - information about files output from DQM, their location, type and internal format.
\item Input data file - any directory and file naming conventions and details as well as scripts that use this information
\item Input data format - internal format of an input data file
\item POMS configuration files - means of access to and editing of POMS requests.
\item Reduced data format - location of reduced data files (if any) and description of their internal format
\item SAM metadata files  - what SAM metadata files are needed and their description.
\item Slow Control data format - data format and name of code module and description of the database attributes and interface for slow control information.
\end{enumerate}
\renewcommand{\theenumi}{\alph{arabic}}

\noindent Conceptual diagram of the workflows and their dependencies is shown in Figure 2. We aim to make sure that the terminology and its presentation is consistent across components, and comments are welcome.


\begin{thebibliography}{1}

\bibitem{docdb1794}
{DUNE DocDB 1794: \textit{ProtoDUNE-SP Technical Design Report }}\\
\url{http://docs.dunescience.org:8080/cgi-bin/ShowDocument?docid=1794}



\bibitem{docdb1212}
{DUNE DocDB 1212: \textit{Design of the Data Management System for the protoDUNE Experiment}}\\
\url{http://docs.dunescience.org:8080/cgi-bin/ShowDocument?docid=1212}

\bibitem{eos}
{The CERN Exabyte Scale Storage}\\
\url{http://information-technology.web.cern.ch/services/eos-service}


\bibitem{fts}
{The Fermilab File Transfer System}\\
\url{http://cd-docdb.fnal.gov/cgi-bin/RetrieveFile?docid=5412&filename=datamanagement-changeprocedures.pdf&version=1}


\bibitem{docdb1811}
{DUNE DocDB 1811: \textit{Prompt Processing System Requirements for the Single-Phase protoDUNE}}\\
\url{http://docs.dunescience.org:8080/cgi-bin/ShowDocument?docid=1811}

\bibitem{p3s}
{A Design of the Prompt Processing System for the Single-Phase protoDUNE experiment (NP04 )}\\
\url{http://docs.dunescience.org:8080/cgi-bin/ShowDocument?docid=1861}

\bibitem{docdb2089}
{Proposed Initial Data Reduction for protoDUNE/SP}\\
\url{http://docs.dunescience.org/cgi-bin/RetrieveFile?docid=2089}

\bibitem{poms}
{Production Operations Management Service (POMS)}\\
\url{https://cdcvs.fnal.gov/redmine/projects/prod_mgmt_db}


\bibitem{lxbatch}
{The CERN batch computing service}\\
\url{http://information-technology.web.cern.ch/services/batch}



\bibitem{neut}
{Neutrino Computing Cluster at CERN}\\
\url{https://twiki.cern.ch/twiki/bin/view/CENF/NeutrinoClusterCERN}




\end{thebibliography}


\end{document}

%%% Local Variables:
%%% mode: latex
%%% TeX-master: t
%%% End:
\grid

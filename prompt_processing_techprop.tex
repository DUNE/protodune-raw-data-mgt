\documentclass[pdftex,12pt,letter]{article}
\usepackage[margin=0.75in]{geometry}
\usepackage{verbatim}
\usepackage{graphicx}
\usepackage{xspace}
\usepackage{cite}
\usepackage{url}
\usepackage[pdftex,pdfpagelabels,bookmarks,hyperindex,hyperfigures]{hyperref}

\newcommand{\pd}{protoDUNE\xspace}
%\newcommand{\pdsp}{pD/SP\xspace}
\newcommand{\xrd}{XRootD\xspace}
\newcommand{\expname}{\textit{NP04}\xspace}

\title{Technical Proposal for Prompt Processing System in the Single-Phase \pd}
\date{\today}
\author{M.\,Potekhin and B.\,Viren}


\begin{document}
\maketitle

\begin{abstract}
\noindent  This note describes a proposal for the design of
the prompt processing system in the Single-Phase \pd
(CERN experiment \expname). We start with a brief summary of the data characteristics and data handling patterns
in \pd (DocDB\,1086 \cite{docdb1086}, DocDB\,1212 \cite{docdb1212}) and present a design which satisfies the basic requirements set
forth in  DocDB 1811 \cite{docdb1811}.
\end{abstract}

%%%%%%%%%%%%%
\section{Overview}
\subsection{Data Scenarios and Data Handling}
\label{sec:rawdata}
Scenarios for data taking in \pd are collected and maintained in \cite{docdb1086}. The following parameters
are assumed at the time of writing: the raw data rate will 1.4\,GB/s after lossless compression, and to size up
the system and account for contingencies rates up to 3\,GB/s are under consideration.

Outline of the design of the data handling system is documented in  \cite{docdb1212} and other prior \pd documentation.
According to it, the data is tranmitted from the online buffer to the CERN EOS via a 20 Gbps full-duplex network connection.
The EOS system serves as the hub and staging area for the \pd raw data from which
it gets committed to the tape archive at CERN and also transmitted to FNAL and potentially
other US and international locations. All or most links in the data transmission chain will be based on F-FTS.

\subsection{Outline of Prompt Processing}
\label{sec:outline}
\subsubsection{Motivation}
According to \cite{docdb1811}  prompt processing needs to occur on the scale
of tens of minutes (or better) from the time the data was taken. It's purpose is to
provide a more in-depth data QA and assessment of the detector operating conditions
than is afforded by most basic histogrammin e.g. of the ADC counts.


\subsubsection{Staging Data for Prompt Processing}
CERN EOS appears to be the best suited platform from which to
serve the data for prompt processing for the following reasons:
\begin{itemize}

\item The online buffer is likely to operate at a significant data rate (see above) and additional I/O load on the buffer is undesirable
due to practical limits on the storage bandwidth and in order to guarantee stability of operation for the online buffer and DAQ in general.

\item It can be be expected that the data arrives to EOS rather quickly after having been captured in the online buffer (e.g.$\sim$1\,min) so
the latency due to this transfer is acceptable.

\item EOS has variety of interfaces including \xrd which simplifies access from various types of locations (both inside and outside the CERN perimeter).

\end{itemize}


\subsection{Content of the Prompt Processing}
This is a compressed summary of information presented in \cite{docdb1811}.
For more detailed explanation please see the source.

\subsubsection{Data Processing}
 The categories are as follows:

\begin{description}

\item[DAQ] A summary of DAQ-level data (with no decompression) such as summaries of data
 rates  or summaries of any metadata, status codes the provided by the DAQ etc.
This processing is a candidate for running directly within artDAQ
monitor processes instead of prompt-processing and is included here
for completeness.

\item[ADC] A summary of ADC-level data e.g. mean/RMS (requires data decompression).


\item[FFT] A summary of the ADC-level data in frequency space. It requires running a discrete Fourier
transform (FFT) on channel waveforms. This largely  provides measures of noise and its
evolution.

\item[Sig] A summary of the data \textit{after signal processing}.
The processing is in  frequency space and so uses the output of FFT.  It includes software
noise subtraction, filtering and deconvolution of the response function.

\item[Reco] Results from running some type of reconstruction (perhaps simplified).
It may, for  example, provide a coarse count of straight muon track candidates.

\end{description}

\subsubsection{Visualization}
The \textit{Visualization} category of processing
 may take data output from any of the above listed stages in
order to efficiently present it to the end-user. 
% This stage will need to
%have a sampling fraction  based on both the processing requirements
%(e.g. processing time) and based on how fast a human can absorb and understand
%the information as it undergoes updates.
It may include items such as:

\begin{itemize}

\item Histograms of statistical quantities.

\item Strip charts showing their history.

\item Various statistics dynamically updated over some some fixed time window.

\item 2D displays of underlying values such as spectrograms of the \textit{FFT}
  output (vs wire), or time vs wire using output from \textit{ADC} and \textit{Sig}.

\end{itemize}

\subsubsection{Other Applications of the Data Processing Components}
Results of the intermediate stages of Data Processing may not need to be preserved.
However, it is worth noting that there is a potential for large processing efficiency
gains to be made if the entire data can be run through the \textit{Sig} processing followed by ZS.

\section{Requirements}

\subsection{Workflow Automation}

Prompt processing represents a use case quite different from managed production or user analysis, and is closer to
procesing data in streaming mode.

Based on the information above the prompt processing must be driven by the data arriving from DAQ as opposed
to user intervention, i.e. have a high degree of automation. This also applies to the tiered structure of processing as outlined above,
i.e. the ability to select a fractional sample of the data stream as is progresses through the chain, at every stage.
For example, if 100 cores are allocated to FFT calculation this may be done for approximately 10\% of the events
in streaming mode. Depending on the type of deconvolution being employed (1D vs 2D) proceeding to the next
step will result in further factor of 10 reduction in the number of events that can be processed at this scale.
Event reconstruction will require additional CPU and it may be necessary to scale down the number of
events at this stage.

While the arguments presented above are quite preliminary, it is likely that it will be necessary to automate
the workflow such that a configurable fraction of data produced in each tier of processing is ingested by the next stage.

The automation must allow some flexibility in setting what
sampling fractions are employed.  At least these
fractions will need to be specified on a per run period basis.  Given
that some jobs' processing times depend on the data content there may
be a need for a more dynamic adjustment of the sampling fractions, perhaps
automated so as to avoid backlogs and provide optimal throughput of the
overall system.

\subsection{Resource Utilization}
The system is required to be flexible enough to take advantage of varying type of resources. For example, the ``neut'' cluster at CERN
is a rather vanilla example of a HTCondor installation. It will be scaled to up to 350 nodes with multiple cores each. It does not have Grid middleware
installed and technically this should not be necessary since it's local to the data. This implies that it must be possible to generate jobs to
run on this cluster locally.

Flexibility needs to be retained to utilize the lxbatch facility at CERN or even run jobs on the facilities at FNAL, BNL and other such locations
in the US and Europe. This is possible by utilizing EOS interfaces such as XRootD.

As mentioned above, it may be required to explore computing resources
at other institutions.  Such explorations must consider the nominal
turn around time that makes this processing ``prompt''.

\subsection{Monitoring}
It will be neccesary to monitor the prompt processing system at a few
levels, from general availability and throughput of the computing
element to individual job level and log file information.  A web
service will be required to ensure optimal and user-friendly interface
to the monitoring system.

Another web service (potentially integrated with the above) is
required to display results of the \textit{Vis} stage for use by
detector experts, commissioning and operations shift workers as well
as general collaborators who may not be at CERN.  Some requirements
for this web service(s) are:

\begin{itemize}
\item Display current and past sets of \textit{Vis} stage outputs.
\item Automatically (``live'') update key \textit{Vis} results as they become available.
\item Provide features to navigate to more detailed visualizations based on summaries.
\end{itemize}


\subsection{Interfaces}
The prompt processing system will need to interface with the data handling system (\textit{ie} F-FTS/SAM). A simple example
would be prevention of the data being purged from EOS while it's still needed for processing.

One or more web services, described above, will need access to
information about the prompt-processing itself as well as the output
of the \textit{Vis} stage.


\begin{thebibliography}{1}
\bibitem{docdb1086}
{DUNE DocDB 1086: \textit{ protoDUNE/SP data scenarios with full stream (spreadsheet)}}\\
\url{http://docs.dunescience.org:8080/cgi-bin/ShowDocument?docid=1086}

%\bibitem{docdb186}
%{DUNE DocDB 186: \textit{ ProtoDUNE Proposal}}\\
%\url{http://docs.dunescience.org:8080/cgi-bin/ShowDocument?docid=186}


%\bibitem{docdb1209}
%{DUNE DocDB 1209: \textit{Basic Requirements for the protoDUNE Raw Data Mangement System}}\\
%\url{http://docs.dunescience.org:8080/cgi-bin/ShowDocument?docid=1209}


\bibitem{docdb1212}
{DUNE DocDB 1212: \textit{Design of the Data Management System for the protoDUNE Experiment}}\\
\url{http://docs.dunescience.org:8080/cgi-bin/ShowDocument?docid=1212}

\bibitem{docdb1811}
{DUNE DocDB 1811: \textit{Prompt Processing System Requirements for the Single-Phase protoDUNE}}\\
\url{http://docs.dunescience.org:8080/cgi-bin/ShowDocument?docid=1811}


\end{thebibliography}


\end{document}
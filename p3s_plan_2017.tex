\documentclass[pdftex,12pt,letter]{article}
\usepackage[binary-units=true]{siunitx}
\usepackage[margin=0.75in]{geometry}
\usepackage{verbatim}
\usepackage{graphicx}
\usepackage{cite}
\usepackage{color}
\usepackage[pdftex,pdfpagelabels,bookmarks,hyperindex,hyperfigures]{hyperref}
\usepackage{xspace}
%\usepackage[firstpage]{draftwatermark}


\bibliographystyle{unsrt}

\newcommand{\fixme}[1]{\textbf{FIXME: #1}}    
\newcommand{\pd}{protoDUNE\xspace}

\title{Planning of the \pd Prompt Processing System}
\date{\today}
\author{M.Potekhin, B. Viren}

\begin{document}
%\SetWatermarkText{DRAFT}
%\SetWatermarkLightness{0.9}
%\SetWatermarkScale{3}

\maketitle

\begin{abstract}
\noindent The purpose of this document is to inform the leadership
of the Single-Phase \pd experiment (NA04) about the scope, deliverables,
schedules, interfaces and other crucial characteristics of the prompt processing
system (p3s).
\end{abstract}

% \tableofcontents

\pagebreak

\section{Overview}
We will use the name ``p3s'' to refer to the \textit{\pd Prompt Processing System.}
The primary purpose of p3s is to support \textit{Data Quality Monitoring} (DQM) in \pd.
The following шtems will be covered in this document:
\begin{itemize}
\item p3s scope and deliverables
\item minimum required capabilities and functionality of p3s at key points in time 

\item a set of milestones in about 2 months time increments, including relevant testing

\item resource assessment including the effort profile in FTEs and people assigned to the task

\item outline of risk assessment 
\end{itemize}


\section{The scope of p3s}
\subsection{The p3s data input and output}
It has been decided that the boundary between the DAQ and other online and
offline systems
(i.e. data handling and prompt processing) exists in the form of the Online Buffer
which is attached to DAQ. 
The ``\pd/SP Data Scenarios'' spreadsheet~\cite{docdb1086}
describes  possible running conditions and estimates for their
resulting data volumes and rates and interpretations in terms of
network and disk bandwidth. 

Data is written to the buffer as files containing
necessary header information and are assumed to be ready to be committed to mass
storage (disk and tape) without further modifications. Most of data handling will
be done using F-FTS \cite{docdb1212,fts}. The p3s may read these data from
either of the two locations:
\begin{itemize}
\item the Online Buffer itself (via a network protocol such as XRootD~\cite{xrootd})
\item CERN EOS~\cite{eos}
\end{itemize}

\noindent Calculations done in p3s 
will result in some sort of a ``visual product'', for example a histogram, plot, event display
or a summary table in order to provide information to the experiment operators in a timely manner
via a Web interface and if necessary also as data stored as files. 
Since the volume of visual and numeric informaiton pertinent to DQM may be large and hard
to quickly comprehend, there will be automated alerts to the operators when certain parameters
are outside of their nominal range.

A portion of the data produced by p3s will
be copied to mass storage for preservation and distributed to the Collaboration for
bookkeeping and operations support purposes.


\subsection{p3s and the Online Monitoring as complementary systems}
The p3s fills the gap between the very low latency Online Monitoring (OM)
system which is engineered as a component of DAQ on one hand, and
the general offline processing on the other. Conceptually, they both serve
the same purpose of generating time-critical DQM information on a short time
scale needed to ascertain the condition and performance of both the detector
and the DAQ and are complementary to each other. 
The benchmark turnaround time for p3s jobs 
according to the requirements set forth in a separate document 
\cite{docdb1811}  is set at 10\,min for reference purposes,
of course there will be several jobs types and execution times will
vary for many reasons. This time scale fulfills the need to have \textit{actionable}
DQM information in time for operators to take action and prevent loss
of useable data and/or valuable beam time.

By having p3s as a separate
component in addition to the OM the following issues are adressed:
\begin{itemize}

\item \textbf{scalability}: DAQ and OM operate in a purpose-built computing environment characterized
by an extremely high bandwidth and a relatively limited CPU budget, as dictated by the
primary mission of these systems. Scaling up computational capabilities of DAQ
is still possible but likely to be prohibitevely expensive due to nature and cost of
the hardware used in DAQ. The prompt processing system from the very beginning
was conceptualized as built upon common (and in fact recycled) general purpose hardware
with ample CPU and somewhat limited network bandwidth. This makes scaling out of
the system very economical and even allows for inclusion of opportunistic resources
with very few requirements to be met.

\item \textbf{test of data integrity}: OM will consume data in the form of network streams generated
by the DAQ specifically for that purpose, i.e.~OM won't operate on actual \textit{files} generated
by the DAQ. This achieves a degree of simplicity of data handling and very low latency
in distributing the data to CPU, but also precludes the possibility to ascertain that the data
in its final form (i.e.~as finalized files on disk prepared for sinking to mass storage) is indeed
formatted correctly and is otherwise valid in that shape. The p3s whose input is precisely the finished
files coming out of DAQ serves in that role in addition to others.

\item \textbf{coupling/dependency}: 
The p3s almost entirely independent of DAQ since it only has to comply with the format
of the raw data file which should be relatively stable
throughout the data taking period. Payloads running in p3s can be quickly modified
and adapted to new conditions and requirements during \pd commissioning and
data taking periods without impacting operation of DAQ and its most immediate
monioring needs (fulfilled by the OM).

\item \textbf{access to data and monitoring software}:
It is expected that access to DAQ, OM and monitoring data generated in the latter will be
restricted to a very limited number of people to reduce operational risks 
(e.g.~it is undesirable to have a hundred users browsing histograms coming out of OM
or introduce new algorithms into the online monitoring). Being more scalable and 
decoupled from the core DAQ the p3s will allow to make visual and data products
more widely accessible to the Collaboration and enable more users to contribute
to creating and modifying payloads ``in situ'' as the experiment progresses.


\end{itemize}



\subsection{Categories of Jobs in p3s}
\label{sec:categories}

This is a compressed summary of information presented in \cite{docdb1811}.
For now we focus only on TPC Data Processing with the Photon Detector Data
to be considered at a later date. The categories are as follows:

\begin{description}

\item[DAQ (no data decompression)] A summary of DAQ-level data  such as summaries of data
 rates  or summaries of any metadata, status codes the provided by the DAQ etc.
Most of this functionality will exist in OM so this is optional in p3s, with may perhaps
provide additional level of detail and logging.

\item[ADC (requires data decompression)] A summary of ADC-level data e.g. mean/RMS
values at channel level and as a statistics over various groupings and level of detail
(ASIC, FEMB, RCE, APA etc).


\item[FFT] A summary of the ADC-level data in frequency space. It requires running a discrete Fourier
transform (FFT) on channel waveforms. This largely  provides measures of noise and its
evolution.

\item[SIG] A summary of the data \textit{after signal processing}.
The processing is in  both time domain and in frequency domain (so uses the output of FFT).
It includes items such as
\begin{itemize}
\item ``stuck code'' mitigation
\item coherent noise removal
\item noise subtraction and filtering
\item deconvolution of the response function
\item calculation of signal correlations for diagnostic purposes
\end{itemize}

\item[RECO] Results from running some type of reconstruction (perhaps greatly simplified).
It may, for  example, provide a coarse count of straight muon track candidates.

\end{description}

\subsection{The Hardware}
It is anticipated that only a limited amount of new hardware will need to be procured
for p3s. The hardware base of the system will mainly consist of the compute servers
which currently belong to the \textit{neut} cluster of the CERN Neutrino Platform
\cite{neut}
and its DQM partition named \textit{neutdqm}. Other facilities such as \textit{lxbatch}
\cite{lxbatch}
and others may be added if and when they become available since p3s is a pilot-based
system which facilitates federation of resources, and is batch-system-agnostic.

In addition to compute servers, a small number Web and DB servers will be necessary
which possess adequate performance characteristics. Storage requirements are being
scoped out at this point in time.



\section{Required Capabilities}
\label{sec:capabilities}

The p3s must have capabilties to support categories of processing outlined
in Sec.\,\ref{sec:categories}. The term ``task'' is used here to designate a set
of related jobs, for a example a processing chain where each job consumes
data produced by its predecessor. The p3s supports description of tasks as DAGs
and also supports task templates for ease of operation.
 The first job in the chain reads data from
the Online Buffer or EOS. In the limiting case, a task may contain just one payload
job. Submission of individual \textit{ad hoc} jobs by users is also supported.

\begin{description}
\item[Job and Task Management]\ 
\begin{itemize}

\item automatic generation of tasks upon arrival of fresh data to the buffer

\item distribution of jobs to processing resources assigned to p3s

\item management of tasks i.e.~orchestration of execution of jobs within
a task according to the dependencies between jobs

\item prioritization of tasks and jobs (during assignmenet to processing slots)
 including automated dynamic changes of priorities in order to ensure completion
of critical tasks

\end{itemize} 

\item[User and System interfaces]\ 
\begin{itemize}

\item full remote control of the system by operators via appropriate interfaces

\item interface to the Online Buffer and EOS in order to access the input data

\item interface to F-FTS necessary to move a portion of p3s output to external
sites and mass storage

\item Web interface for task monitoring and debugging of p3s


\end{itemize} 

\item[Web Access to Data]\ 
\begin{itemize}

\item the data produced by p3s will be made available to the users via a Web service
(e.g. a Web page populated with plots selected according to user-specified criteria)

\item for the data which is preserved as data files (as opposed to visual products)
there will be a catalog accessible via a browser/Web page

\end{itemize} 

\end{description}

\section{Components and Deliverables}

The core of p3s is a Django-based \cite{django} application written in Python. The
deliverables in the systems category are as follows:

\begin{description}
\item[p3s server] --- a Web service which supports task management capabilities as per Sec.\,\ref{sec:capabilities}

\item[user tools] --- such as CLI clients necessary to manipulate the state of the system, submit
and manage tasks in the manual mode if necessary, and to perform general management and
maintenance of the system

\item[data server] --- a Web/network service which makes visual and data products produced by p3s
available to the users

\item[data network access] --- a XRootD-based service continuously running on the DQM cluster
and communicating with small-footprint XRootD service running on the Online Buffer, and also EOS.

\item[Web and DB servers] --- needed as a platform to support the items listed above. These
can be based on a variety of products and offer a degree of interoperability.

\end{description}

\noindent 

\section{Timeline and Milestones}

The items below include system deployment, functional and integration testing:

\begin{description}

\item[March'17: p3s dev server] --- a continuously running instance of p3s deployed on the DQM cluster,
utilizing the ``Django development server'' and \textit{sqlite} database back-end. LArSoft payloads as
test case. Visual and data products served from the same server.

\item[April'17: XRootD] --- a continuously running XRootD service deployed on the DQM cluster with
fully functional interface to EOS and ongoing test transfers.

\item[June'17: full chain] --- an instance of p3s running on dev server/sqlite with automated
generation of workflows
\begin{itemize}
\item DAQ vertical slice readiness
\end{itemize}

\item[July'17: Web Service for Visual and Data Products] --- an instance of a Web server optimized
for delivery of visual and data products produced by p3s.

\item[August'17] --- p3s migration to Apache and PostgreSQL. Integration testing with the
DAQ Online Buffer.

\item[September'17] --- p3s/DAQ integration testing
\begin{itemize}
\item Full DAQ test readiness
\end{itemize}

\item[Q4'17] --- p3s stress and scalability testing, additional development

\item[Q1'18] --- continuous p3s operation with realistic workloads, running
MC/Reco jobs in utility mode

\item[Q2'18]\ 
\begin{itemize}
\item Full data taking readiness
\end{itemize}

\end{description}


\section{Resources}

\subsection{Effort Levels}

Development/engineering
\begin{itemize}
\item M.Potekhin (BNL) --- 100\%FTE
\item B.Viren (BNL) --- 10\%FTE
\end{itemize}

\noindent System administration and support

\begin{itemize}
\item N.Benekos (CERN) --- 50\%FTE
\item G.Savage (FNAL) --- 50\%FTE
\end{itemize}

\subsection{Mat\'eriel}
To be useful, p3s needs O(100) cores during its operation. In addition,
a small number of adequately configured machines are needed for
deployment of Web and DB servers to support the p3s function.

It is a working assumption at the time of writing that a significant portion of
the \textit{neut} cluster \cite{neut} will be assigned for p3s use as \pd ramps up
its operations in 2017 and into 2018.


\section{Risk Assessment}

Technologies used in p3s are not new, they are mature and tested in the field, which reduces
overall implementation risk.

\clearpage
\begin{thebibliography}{1}

\bibitem{docdb1086}
{DUNE DocDB 1086: \textit{ protoDUNE/SP data scenarios with full stream (spreadsheet)}}\\
\url{http://docs.dunescience.org:8080/cgi-bin/ShowDocument?docid=1086}




\bibitem{docdb1212}
{DUNE DocDB 1212: \textit{Design of the Data Management System for the protoDUNE Experiment}}\\
\url{http://docs.dunescience.org:8080/cgi-bin/ShowDocument?docid=1212}

\bibitem{fts}
{The Fermilab File Transfer System}\\
\url{http://cd-docdb.fnal.gov/cgi-bin/RetrieveFile?docid=5412&filename=datamanagement-changeprocedures.pdf&version=1}


\bibitem{xrootd}
{XRootD}\\
\url{http://www.xrootd.org}

\bibitem{eos}
{The CERN Exabyte Scale Storage}\\
\url{http://information-technology.web.cern.ch/services/eos-service}



\bibitem{docdb1811}
{DUNE DocDB 1811: \textit{Prompt Processing System Requirements for the Single-Phase protoDUNE}}\\
\url{http://docs.dunescience.org:8080/cgi-bin/ShowDocument?docid=1811}



\bibitem{neut}
{Neutrino Computing Cluster at CERN}\\
\url{https://twiki.cern.ch/twiki/bin/view/CENF/NeutrinoClusterCERN}

\bibitem{lxbatch}
{The CERN batch computing service}\\
\url{http://information-technology.web.cern.ch/services/batch}

\bibitem{django}
{Django}\\
\url{https://docs.djangoproject.com/en/1.10/}


\end{thebibliography}


\end{document}

%%% Local Variables:
%%% mode: latex
%%% TeX-master: t
%%% End:

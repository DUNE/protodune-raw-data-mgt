\documentclass[pdftex,12pt,letter]{article}
\usepackage[binary-units=true]{siunitx}
\usepackage[margin=0.75in]{geometry}
\usepackage{verbatim}
\usepackage{graphicx}
\usepackage{cite}
\usepackage{color}
\usepackage[pdftex,pdfpagelabels,bookmarks,hyperindex,hyperfigures]{hyperref}
\usepackage{xspace}
%\usepackage[firstpage]{draftwatermark}


\bibliographystyle{unsrt}

\newcommand{\fixme}[1]{\textbf{FIXME: #1}}    
\newcommand{\pd}{protoDUNE\xspace}

\title{Planning of the \pd Prompt Processing System}
\date{\today}
\author{M.Potekhin, B. Viren}

\begin{document}
%\SetWatermarkText{DRAFT}
%\SetWatermarkLightness{0.9}
%\SetWatermarkScale{3}

\maketitle

\begin{abstract}
\noindent The purpose of this document is to inform the leadership
of the Single-Phase \pd experiment (NA04) about the scope, deliverables,
schedules, interfaces and other crucial characteristics of the prompt processing
system (p3s).
\end{abstract}

% \tableofcontents

\pagebreak

\section{Overview}
We will use the name ``p3s'' to refer to the \textit{\pd Prompt Processing System.}
The primary purpose of p3s is to support \textit{Data Quality Monitoring} (DQM) in \pd.
The following шtems will be covered in this document:
\begin{itemize}
\item p3s scope and deliverables
\item minimum required capabilities and functionality of p3s at key points in time 

\item a set of milestones in about 2 months time increments, including relevant testing

\item resource assessment including the effort profile in FTEs and people assigned to the task

\item outline of risk assessment 
\end{itemize}


\section{The scope of p3s}
\subsection{The p3s data input and output}
It has been decided that the boundary between the DAQ and other online and
offline systems
(i.e. data handling and prompt processing) exists in the form of the Online Buffer
which is attached to DAQ. 
The ``\pd/SP Data Scenarios'' spreadsheet~\cite{docdb1086}
describes  possible running conditions and estimates for their
resulting data volumes and rates and interpretations in terms of
network and disk bandwidth. 

Data is written to the buffer as files containing
necessary header information and are assumed to be ready to be committed to mass
storage (disk and tape) without further modifications. Most of data handling will
be done using F-FTS \cite{docdb1212,fts}. The p3s may read these data from
either of the two locations:
\begin{itemize}
\item the Online Buffer itself (via a network protocol such as XRootD~\cite{xrootd})
\item CERN EOS~\cite{eos}
\end{itemize}

\noindent Calculations done in p3s 
will result in some sort of a ``visual product'', for example a histogram, plot, event display
or a summary table in order to provide information to the experiment operators in a timely manner
via a Web interface and if necessary also as data stored as files. 
Since the volume of visual and numeric informaiton pertinent to DQM may be large and hard
to quickly comprehend, there will be automated alerts to the operators when certain parameters
are outside of their nominal range.

A portion of the data produced by p3s will
be copied to mass storage for preservation and distributed to the Collaboration for
bookkeeping and operations support purposes.


\subsection{p3s and the Online Monitoring as complementary systems}
The p3s fills the gap between the very low latency Online Monitoring (OM)
system which is engineered as a component of DAQ on one hand, and
the general offline processing on the other. Conceptually, they both serve
the same purpose of generating time-critical DQM information on a short time
scale needed to ascertain the condition and performance of both the detector
and the DAQ and are complementary to each other. 
The benchmark turnaround time for p3s jobs 
according to the requirements set forth in a separate document 
\cite{docdb1811}  is set at 10\,min for reference purposes,
of course there will be several jobs types and execution times will
vary for many reasons. This time scale fulfills the need to have \textit{actionable}
DQM information in time for operators to take action and prevent loss
of useable data and/or valuable beam time.

By having p3s as a separate
component in addition to the OM the following issues are adressed:
\begin{itemize}

\item \textbf{scalability}: DAQ and OM operate in a purpose-built computing environment characterized
by an extremely high bandwidth and a relatively limited CPU budget, as dictated by the
primary mission of these systems. Scaling up computational capabilities of DAQ
is still possible but likely to be prohibitevely expensive due to nature and cost of
the hardware used in DAQ. The prompt processing system from the very beginning
was conceptualized as built upon common (and in fact recycled) general purpose hardware
with ample CPU and somewhat limited network bandwidth. This makes scaling out of
the system very economical and even allows for inclusion of opportunistic resources
with very few requirements to be met.

\item \textbf{test of data integrity}: OM will consume data in the form of network streams generated
by the DAQ specifically for that purpose, i.e.~OM won't operate on actual \textit{files} generated
by the DAQ. This achieves a degree of simplicity of data handling and very low latency
in distributing the data to CPU, but also precludes the possibility to ascertain that the data
in its final form (i.e.~as finalized files on disk prepared for sinking to mass storage) is indeed
formatted correctly and is otherwise valid in that shape. The p3s whose input is precisely the finished
files coming out of DAQ serves in that role in addition to others.

\item \textbf{coupling/dependency}: 
The p3s almost entirely independent of DAQ since it only has to comply with the format
of the raw data file which should be relatively stable
throughout the data taking period. Payloads running in p3s can be quickly modified
and adapted to new conditions and requirements during \pd commissioning and
data taking periods without impacting operation of DAQ and its most immediate
monioring needs (fulfilled by the OM).

\item \textbf{access to data and monitoring software}:
It is expected that access to DAQ, OM and monitoring data generated in the latter will be
restricted to a very limited number of people to reduce operational risks 
(e.g.~it is undesirable to have a hundred users browsing histograms coming out of OM
or introduce new algorithms into the online monitoring). Being more scalable and 
decoupled from the core DAQ the p3s will allow to make visual and data products
more widely accessible to the Collaboration and enable more users to contribute
to creating and modifying payloads ``in situ'' as the experiment progresses.


\end{itemize}



\subsection{Categories of Jobs in p3s}
\label{sec:scope}

\label{sec:categories}
This is a compressed summary of information presented in \cite{docdb1811}.
For now we focus only on TPC Data Processing with the Photon Detector Data
to be considered at a later date. The categories are as follows:

\begin{description}

\item[DAQ (no data decompression)] A summary of DAQ-level data  such as summaries of data
 rates  or summaries of any metadata, status codes the provided by the DAQ etc.
Most of this functionality will exist in OM so this is optional in p3s, with may perhaps
provide additional level of detail and logging.

\item[ADC (requires data decompression)] A summary of ADC-level data e.g. mean/RMS
values at channel level and as a statistics over various groupings and level of detail
(ASIC, FEMB, RCE, APA etc).


\item[FFT] A summary of the ADC-level data in frequency space. It requires running a discrete Fourier
transform (FFT) on channel waveforms. This largely  provides measures of noise and its
evolution.

\item[SIG] A summary of the data \textit{after signal processing}.
The processing is in  both time domain and in frequency domain (so uses the output of FFT).
It includes items such as
\begin{itemize}
\item ``stuck code'' mitigation
\item coherent noise removal
\item noise subtraction and filtering
\item deconvolution of the response function
\item calculation of signal correlations for diagnostic purposes
\end{itemize}

\item[RECO] Results from running some type of reconstruction (perhaps greatly simplified).
It may, for  example, provide a coarse count of straight muon track candidates.

\end{description}

\section{Deliverables}


\clearpage
\begin{thebibliography}{1}

\bibitem{docdb1086}
{DUNE DocDB 1086: \textit{ protoDUNE/SP data scenarios with full stream (spreadsheet)}}\\
\url{http://docs.dunescience.org:8080/cgi-bin/ShowDocument?docid=1086}




\bibitem{docdb1212}
{DUNE DocDB 1212: \textit{Design of the Data Management System for the protoDUNE Experiment}}\\
\url{http://docs.dunescience.org:8080/cgi-bin/ShowDocument?docid=1212}

\bibitem{fts}
{The Fermilab File Transfer System}\\
\url{http://cd-docdb.fnal.gov/cgi-bin/RetrieveFile?docid=5412&filename=datamanagement-changeprocedures.pdf&version=1}


\bibitem{xrootd}
{XRootD}\\
\url{http://www.xrootd.org}

\bibitem{eos}
{The CERN Exabyte Scale Storage}\\
\url{http://information-technology.web.cern.ch/services/eos-service}



\bibitem{docdb1811}
{DUNE DocDB 1811: \textit{Prompt Processing System Requirements for the Single-Phase protoDUNE}}\\
\url{http://docs.dunescience.org:8080/cgi-bin/ShowDocument?docid=1811}



\bibitem{neut}
{Neutrino Computing Cluster at CERN}\\
\url{https://twiki.cern.ch/twiki/bin/view/CENF/NeutrinoClusterCERN}

\bibitem{lxbatch}
{The CERN batch computing service}\\
\url{http://information-technology.web.cern.ch/services/batch}


\bibitem{htcondor}
{HTCondor}
\url{https://research.cs.wisc.edu/htcondor/}

\bibitem{arda}
{The CERN Dashboard project}\\
CERN-IT-NOTE-2007-048) and \url{http://dashboard.cern.ch/}

\bibitem{root}
{ROOT}\\
\url{https://root.cern.ch/}


\bibitem{larsoft}
{The Liquid Argon Software (LArSoft) Collaboration}\\
\url{http://larsoft.org/}

\bibitem{fife}
{Fabric for Intensity Frontier Experiments}\\
\url{http://fife.fnal.gov/}



\end{thebibliography}


\end{document}

%%% Local Variables:
%%% mode: latex
%%% TeX-master: t
%%% End:

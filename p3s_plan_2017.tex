\documentclass[pdftex,12pt,letter]{article}
\usepackage[binary-units=true]{siunitx}
\usepackage[margin=0.75in]{geometry}
\usepackage{verbatim}
\usepackage{graphicx}
\usepackage{cite}
\usepackage{color}
\usepackage[pdftex,pdfpagelabels,bookmarks,hyperindex,hyperfigures]{hyperref}
\usepackage{xspace}
\usepackage{amssymb}

%\usepackage[firstpage]{draftwatermark}


\bibliographystyle{unsrt}

\newcommand{\fixme}[1]{\textbf{FIXME: #1}}    
\newcommand{\pd}{protoDUNE\xspace}

\title{Planning of the \pd Prompt Processing System}
\date{\today}
\author{M.Potekhin, B. Viren}

\begin{document}
%\SetWatermarkText{DRAFT}
%\SetWatermarkLightness{0.9}
%\SetWatermarkScale{3}

\maketitle

\begin{abstract}
  \noindent The purpose of this document is to inform the leadership
  of the Single-Phase \pd experiment (NA04) about the scope,
  deliverables, schedules, interfaces and other crucial
  characteristics of the prompt processing (PP).
\end{abstract}

\tableofcontents

\pagebreak

\section{Overview}
It is important to monitor the data being produced in the Single-Phase
\pd experiment in a way that allows fast identification and correction
of any problems.  Monitoring algorithms can require large volumes of
data or large CPU resources.  Algorithms that require both extremes
require more hardware than available or can not return results quickly
enough to be actionable.  This leads to recognizing two opposing ends
of a spectrum of possible monitoring jobs.  Including also full
production (FP) processing there are three categories:

\begin{description}
\item[OM] \textit{Online Monitoring} consists of fast algorithms consuming large fractions of raw data and running directly in the special distributed and asynchronous DAQ computing environment.
\item[PP] \textit{Prompt Processing} consists of relatively long-running algorithms consuming small subsamples of the raw data and running in a conventional computing environment with a special job management system.
\item[FP] \textit{Full Production} consists any excess noise filtering, signal processing, activity imaging, object formation, pattern recognition and other reconstruction.
\end{description}

These three categories are overlapping in terms of which algorithms
can or will run in each.  To efficiently use our limited software
development effort it is important to find ways to write the needed
algorithms in a way that allow their reuse in the three contexts.
Because the code for all three categories are ultimately based on
\textit{art} this reuse can be achieved with relatively modest
effort.\footnote{The authors recommend that algorithm developers
  embrace the new ``tool'' feature provided by the latest version of
  \textit{art}.}  In particular, this should allow OM and PP groups
flexibility in terms of easily moving algorithms from one environment
to another as we better understand our hardware resources and
monitoring needs.  This balance will likely be struck after we have
developed and bench marked the monitoring algorithms.  It is expected
that these benchmarks will be evaluated and algorithms selected in
order that each category of jobs satisfies the very rough requirements
illustrated in table~\ref{tab:contexts}.

\begin{table}[h]
  \centering
  \begin{tabular}[h]{r|ccc}
    & triggers & avail & result \\
    & processed &  CPU & latency \\
    \hline
    OM: & $\sim 10$\% & tens & 1 min \\
    PP: & $\sim 1$\% & hundreds & 10 min \\
    FP: & 100\% & thousands & months \\
  \end{tabular}
  \caption{Qualitative comparison and rough expectations for coverage, resources and performance of online monitoring (OM), prompt processing (PP) and full production (FP) categories of jobs.}
  \label{tab:contexts}
\end{table}


With the above in mind, this note focuses on the unique parts that
make up PP.  It also describes ways in which the PP group intends to
work with the OM and FP groups in order to best make use of our
limited and shared effort in the collaboration.

The PP effort is centered around developing the \textit{\pd Prompt
  Processing System} (p3s) and it is the main deliverable.  At its
heart, the p3s is a special job management system which provides
latency assurances for job completion using a fixed amount of
computing resources at the expense of limiting the data coverage.
This is opposite from traditional job management systems that
emphasize processing 100\% of all input data but do not limit the
duration of that processing.  The scope of p3s is describe further in
Sec.~\ref{sec:p3sscope}.

While the primary focus of the PP group is to provide p3s, the
\textit{payload} jobs that it runs are of great importance as they
provide the actual data quality monitoring (DQM).  The scope that the
PP group assumes related to development of payload jobs is discussed
in Sec.~\ref{sec:categories}.

Guidance on requirements for hardware are given in
Sec.~\ref{sec:hardware} followed by a timeline and milestones in
Sec.~\ref{sec:timeline}, a summary of resources in
Sec.~\ref{sec:resources} and ending with a risk assessment in
Sec.~\ref{sec:risk}.



\section{p3s}
\label{sec:p3sscope}
The \textit{\pd Prompt Processing System} (p3s) is primarily a
platform for management of automated processing of data streams which
gives guarantees of latency of results, if not coverage nor
throughput.  It is an infrastructure element and does not, as a
deliverable, include the actual software for its computational
payloads (which are discussed in Sec.~\ref{sec:categories}).

A related but separate part of the PP deliverable is the p3s
\textit{visualization system}.  This system is intended to display
graphical (eg, histograms, plots, event displays) and other summary
results produced by the payload jobs run by p3s.  This portion of the
system is still conceptual and there is hope that it can be realized
using the same system needed for presenting OM results.  The remainder
of this section will make reference to the visualization system but is
focused on the job management system.


\subsection{Input, Intermediate and Output Files}
\label{sec:io}

From the point of view of raw data flow, the boundary between online
and offline scope is the Online Buffer which is a storage system
attached to DAQ computers.  It exists primarily to assure continued
DAQ operation in the event of an outage of the link to the CERN
network.  The ``\pd/SP Data Scenarios'' spreadsheet~\cite{docdb1086}
describes expected running conditions and provides estimates of data
volumes and rates relevant to this boundary.

All raw files written by the DAQ to the buffer will be transferred to
CERN EOS using the Fermilab file transfer system (F-FTS)
\cite{docdb1212,fts}.  F-FTS takes responsibility for purging any and
all files it is given and can do so in a configurable manner.

The p3s payload jobs are expected to ingest a small percentage
($\lesssim$1\%) of raw data files.  This number is a very rough guess
based on expected CPU requirements of the algorithms, amount of
available CPU and amount of data to provide meaningful feedback about
the data quality.

All files consumed and produced by p3s jobs are assumed to be served
via XRootD~\cite{xrootd}.  This allows flexibility in ingesting the
initial raw data files from one or both of the two possible sources:
\begin{itemize}
\item the Online Buffer
\item CERN EOS~\cite{eos}
\end{itemize}
The Online Buffer, if accessed at all, would only be used as a source
of raw data files.  

The p3s is aware of the input and output files for each job by their
URL (\texttt{root://} or other scheme).  Ultimately, each job is
instructed by p3s, guided by its configuration, what the URLs for
every input file as well as every output file expected.  Any output
file may be used as an input file of a subsequent job.  These
intermediate files connect jobs into an acyclic, directed graph (DAG).

It is work noting that for URLs schemes with a server that allows it,
directly streaming files or even portions of files is allowed.  In the
case of XRootD this can greatly limit the amount of data that must be
transferred in the case where only a single trigger or even a subset
of fragments of one trigger are needed from a raw data file to satisfy
a given payload job.  Jobs which do not have the capability or
inclination to stream must nevertheless provide a way to stage files
in to and out from local storage.  This is expected to be accomplished
by a script which can likely be shared by a variety of core payload
jobs.

The p3s will check for successful job completion and the production of
all expected output files.  Driven by its configured rules, p3s will
purge certain files that made up DAG edges.
As needed, some portion of the files produced by p3s jobs will be
archived to mass storage and made available for distribution to the
collaboration.

Final output files produced by these jobs are also presented to end
users by the p3s \textit{visualization system} in a number of ways as
described below.  The final output files can be categorized in the
following way

\begin{description}
\item[graphics] files in PNG or SVG format representing histograms, plots, event displays, etc.
\item[summary] files in JSON format provides ancillary information for graphics files including captions or other information that may be directly rendered by a user interface.
\item[reference] files which should be retained for browsing by the user and which may include logs or intermediate results. 
\end{description}

\noindent The two end-user interfaces that p3s will provide are:

\begin{description}
\item[web pages] a web server which allows browsing of all results, and include features such as refreshing of specific time critical results
\item[notifications] a system that interprets \textit{summary} files to provide alarm notification via email, mobile push, or other means.
\end{description}

\noindent As mentioned above, the hope is that the visualization needs
of the OM and PP results can be satisfied by a single system.  The PP
group will work with OM and others to achieve that.


\subsection{Required Capabilities}
\label{sec:capabilities}

The p3s must have capabilties to support categories of processing outlined
in Sec.\,\ref{sec:categories}. The term ``task'' is used here to designate a set
of related jobs, for a example a processing chain where each job consumes
data produced by its predecessor. Jobs can be used to add a wide range
of functionalty to the system, e.g.~copy a fraction of data to mass storage
if necessary or produce a set of PNG files based on intermediate ROOT files,
or perhaps generate an alarm if a certain metric falls out of prescribed range.

The p3s supports description of tasks as DAGs and also supports task templates
for ease of operation. The first job in the chain reads data from
the Online Buffer or EOS. In the limiting case, a task may contain just one payload
job; there is no set limit on the number of jobs in the DAG or its topology.
Submission of individual \textit{ad hoc} jobs by users is also supported.

\begin{description}
\item[Job and Task Management]\ 
\begin{itemize}

\item automatic generation of tasks upon arrival of fresh data to the buffer

\item distribution of jobs to processing resources assigned to p3s

\item management of tasks i.e.~orchestration of execution of jobs within
a task according to the dependencies between jobs

\item prioritization of tasks and jobs (during assignmenet to processing slots)
 including automated dynamic changes of priorities in order to ensure completion
of critical tasks

\end{itemize} 

\item[User and System interfaces]\ 
\begin{itemize}

\item full remote control of the system by operators via appropriate interfaces

\item interface to the Online Buffer and EOS in order to access the input data

\item interface to F-FTS necessary to move a portion of p3s output to external
sites and mass storage

\item Web interface for task monitoring and debugging of p3s


\end{itemize} 

\item[Web Access to Results]\ 
\begin{itemize}

\item the data produced by p3s will be made available to the users via a Web service
(e.g. a Web page populated with plots selected according to user-specified criteria)

\item for the data which is preserved as data files (as opposed to visual products)
there will be a catalog accessible via a browser/Web page

\item this item should be shared with OM. 

\end{itemize} 

\item[Notification of Exceptions]\ 
\begin{itemize}

\item summary files will be interpreted and checked for fields holding quantities which are to subject to limits defined in p3s.

\item when a quantity is found to be outside of set limits p3s sends notification to registered individuals. 

\item quantity-experts will have ability to modify the list of watched quantities and their limits.

\end{itemize} 

\end{description}

\subsection{Components and Deliverables}

The core of p3s is a Django-based \cite{django} application written in Python. The
deliverables in the systems category are as follows:

\begin{description}
\item[p3s server] --- a web service which supports the capabilities
  described in Sec.~\ref{sec:capabilities}

\item[user tools] --- such as CLI clients necessary to manipulate the
  state of the system, submit and manage tasks in the manual mode if
  necessary, and to perform general management and maintenance of the
  system.

\item[visualization] --- a web application, ideally shared with OM,
  which presents graphical and other summary results that were
  produced by p3s payload jobs available to end users.

%%% (bv) Do we really need to list this?
% \item[data network access] --- a XRootD-based service continuously
%   running on the DQM cluster and communicating with small-footprint
%   XRootD service running on the Online Buffer, and also EOS.

\item[Web and DB] --- the (software) servers needed as a platform to
  support the items listed above. These can be based on a variety of
  products and offer a degree of interoperability.

\end{description}




\section{Payload Jobs}
\label{sec:categories}

The algorithmic content of the jobs that p3s will run spans a large
range of expertise.  Some jobs will consist of algorithms which may
start out being developed by the OM group but must be moved to PP due
to CPU constraints (and if PP's relative lack of data throughput can
provide enough useful statistics).  On the other end of the spectrum
are jobs which are composed of algorithms that will also run as part
of the \textit{full production} (FP) processing run (and rerun)
offline.

Because of this broad required expertise the PP group can not commit
to providing all payload jobs.  It will however facilitate other
collaborators to develop them.  In particular the group will closely
coordinate with OM and FP groups to identify individuals that can
develop the algorithms.  It will also assist in the development to
help it progress in ways that exploit the benefits of \textit{art} and
realize the flexibility described above.

In the end, the exact makeup of the p3s payload jobs that will run
must be determined by the detector hardware, DAQ and DRA groups and
other stakeholders in monitoring data quality.  However they are
developed, it is any given job will implement one or more of the
following stages which are described more fully in \cite{docdb1811}.
The output of each stage is logically an input to the next although,
due to p3s, there is the potential for parallelization of stages.
Here, this list focuses on TPC data but the photon detection system,
trigger, beam, and cosmic ray information and their correlations
should be covered by the final suite of payload jobs.

\begin{description}

\item[DAQ] category consists of results requiring no data
  decompression and likely will be produced in OM and by p3s jobs.  It
  might includes summaries of things like (compressed) fragment sizes,
  error flags, data rates.

\item[ADC] results require data decompression but no other substantial
  CPU and will consist of summary of ADC-level data.  Examples include
  mean/RMS values over time in various groupings and level of detail
  (ASIC, FEMB, RCE, APA etc).

\item[FFT] category consists of the application of discrete Fourier
  transforms (FFT) to ADC-level data primarily to understand any
  excess noise and its possible evolution.

\item[SIG] includes ADC mitigation, removal of any excess noise and
  the deconvolution of detector response functions in order to
  understand the signals related to actual activity in the detector.

\item[RECO] category includes application of any high-CPU algorithms
  for imaging and reconstructing the topology of activity in the
  detector and may include high level semantic event classifications.

\end{description}

\section{Hardware}
\label{sec:hardware}
The p3s itself will require hardware to run its Web and database
servers and to provide enough storage for intermediate and final
output files as described in section~\ref{sec:io}.  It is expected
that this hardware will be provided by the CERN Neutrino Platform
either from the pool of the machines which beloong to the
existing~\cite{neut} cluster, or procured separately if performance
and scalability tests indicate such necessity.


The storage required for the output files is not currently known.  It
depends on the catalog of payload jobs that will be run and the desire
of the collaboration for the retention of the results.  The largest
single type of result will be event displays.  They will an amount of
storage similar to the raw data from which they are produced.  If one
event display per minute is produced and retained indefinitely some
few tens of TB will be required.  The remanding class of results will
require far less storage.

The hardware required to run the payload jobs themselves of course
depends on the nature of their CPU needs and the fraction of triggers
they consume.  P3s follows a \textit{pilot}-based paradigm and so is
able to provide a uniform execution environment to its payload jobs
over a variety of native hosting environments.  This gives p3s the ability
to scale elastically across different hosts (oor faciltieis) as needed.  Initial expectation
is that DQM jobs will be run by p3s on nodes in the \textit{neutdqm}
partition of \textit{neut} cluster.  Other facilities such as
\textit{lxbatch} \cite{lxbatch} may be added if needed and as available.



\section{Timeline and Milestones}
\label{sec:timeline}
The items below include system deployment, functional and integration testing.
Critical milestones are marked by bullets.

\begin{description}

\item[March'17: p3s dev server] --- a continuously running instance of p3s deployed on the DQM cluster,
utilizing the ``Django development server'' and \textit{sqlite} database back-end. LArSoft payloads as
test case. Visual and data products served from the same server.

\item[April'17: XRootD] --- a continuously running XRootD service deployed on the DQM cluster with
fully functional interface to EOS and ongoing test transfers.

\item[June'17: full chain] --- an instance of p3s running on dev server/sqlite with automated
generation of workflows
\begin{itemize}
\item DAQ vertical slice readiness
\end{itemize}

\item[July'17: Web Service for Visual and Data Products] --- an instance of a Web server optimized
for delivery of visual and data products produced by p3s.

\item[August'17] --- p3s migration to Apache and PostgreSQL. Integration testing with the
DAQ Online Buffer.

\item[September'17] --- p3s/DAQ integration testing
\begin{itemize}
\item Full DAQ test readiness
\end{itemize}

\item[Q4'17] --- p3s stress and scalability testing, additional development

\item[Q1'18] --- continuous p3s operation with realistic workloads, running
MC/Reco jobs in utility mode

\item[Q2'18]\ 
\begin{itemize}
\item Full data taking readiness
\end{itemize}

\end{description}


\section{Resources}
\label{sec:resources}

\subsection{Effort Levels}

Development/engineering
\begin{itemize}
\item M.Potekhin (BNL) --- 100\%\,FTE
\item B.Viren (BNL) --- 10\%\,FTE
\end{itemize}

\noindent System administration and support (working assumption)

\begin{itemize}
\item N.Benekos (CERN) --- 50\%\,FTE (TBD)
\item G.Savage (FNAL) --- 50\%\,FTE (TBD)
\end{itemize}

\noindent The amount of effort needed for development of payload jobs
is not considered in scope but see above sections for discussion of
how the PP group will contribute and assist.

\subsection{Mat\'eriel}

A rough guess is that p3s payload jobs will require O(100) cores
during its operation.  This is an elastic requirement.  More available
cores will allow more intensive processing over a larger percentage of
the raw data while restricting the available hardware can be met by a
scaling back.  This scaling is automatically achieved by p3s.

An small number of adequately configured machines are needed for
deployment of Web and DB servers to support the p3s function.  

It is a working assumption at the time of writing that a significant
portion of the \textit{neut} cluster \cite{neut} will be assigned for
p3s use as \pd ramps up its operations in 2017 and into 2018.  However
and in any case the PP group relies on \pd management to assist in
finding suitable computing resources.


\section{Risk Assessment}
\label{sec:risk}

Technologies used in p3s are not new, they are mature and tested in the field, which reduces
overall implementation risk. The remaining risk factors are
\begin{itemize}

\item insufficient allocation of hardware to p3s jobs can scale down
  the amount of data processed or the capability of the allowed
  algorithms to such an extent that the fraction of triggers or the
  types of features in the data which can be checked leads to a
  failure to identify problems in the detector.

\item lack of coordination and identification of individuals to
  develop the payload jobs which have sufficient coverage and
  performance may likewise lead to failure to identify problems in the
  detector.


\item PP relies on the DAQ group to release updates to the ``data
  access library'' needed for unpacking raw data in a manner prompt
  enough so that p3s payload jobs can adapt, be rebuilt and deployed.
  Failures in this chain can lead to periods of time where detector
  data is unreadable and no data quality monitoring results are
  produced.

\item If the small subsamples of the initial raw data is ingested from
  EOS then PP results depend on the successful and prompt transfer of
  raw data files from the Online Buffer to EOS by F-FTS.  Any delay or
  outage will translate to late or missing PP results.

\item If initial and ongoing expert system administration support is
  not identified for the computers running payload jobs and the
  servers running p3s web service and database then deployment of PP
  may be delayed or unreliable.

\end{itemize}

\clearpage
\begin{thebibliography}{1}

\bibitem{docdb1086}
{DUNE DocDB 1086: \textit{ protoDUNE/SP data scenarios with full stream (spreadsheet)}}\\
\url{http://docs.dunescience.org:8080/cgi-bin/ShowDocument?docid=1086}




\bibitem{docdb1212}
{DUNE DocDB 1212: \textit{Design of the Data Management System for the protoDUNE Experiment}}\\
\url{http://docs.dunescience.org:8080/cgi-bin/ShowDocument?docid=1212}

\bibitem{fts}
{The Fermilab File Transfer System}\\
\url{http://cd-docdb.fnal.gov/cgi-bin/RetrieveFile?docid=5412&filename=datamanagement-changeprocedures.pdf&version=1}


\bibitem{xrootd}
{XRootD}\\
\url{http://www.xrootd.org}

\bibitem{eos}
{The CERN Exabyte Scale Storage}\\
\url{http://information-technology.web.cern.ch/services/eos-service}



\bibitem{docdb1811}
{DUNE DocDB 1811: \textit{Prompt Processing System Requirements for the Single-Phase protoDUNE}}\\
\url{http://docs.dunescience.org:8080/cgi-bin/ShowDocument?docid=1811}



\bibitem{neut}
{Neutrino Computing Cluster at CERN}\\
\url{https://twiki.cern.ch/twiki/bin/view/CENF/NeutrinoClusterCERN}

\bibitem{lxbatch}
{The CERN batch computing service}\\
\url{http://information-technology.web.cern.ch/services/batch}

\bibitem{django}
{Django}\\
\url{https://docs.djangoproject.com/en/1.10/}


\end{thebibliography}


\end{document}

%%% Local Variables:
%%% mode: latex
%%% TeX-master: t
%%% End:
\grid

\documentclass[pdftex,12pt,letter]{article}
\usepackage[margin=0.75in]{geometry}
\usepackage{verbatim}
\usepackage{graphicx}
\usepackage{cite}
\usepackage[pdftex,pdfpagelabels,bookmarks,hyperindex,hyperfigures]{hyperref}
\newcommand{\pd}{protoDUNE\ }
\newcommand{\pdd}{protoDUNE.\ }

\bibliographystyle{unsrt}

\newcommand{\fixme}[1]{\textbf{FIXME: #1}}    


\title{Prototyping and Test of the Single-Phase protoDUNE Buffer Farm}
\date{\today}
\author{M. Potekhin$^a$ and B. Viren$^a$\\
\ \\
$^a$\textit{Brookhaven National Laboratory, Upton NY}
}

\begin{document}
\maketitle

\begin{abstract}
The requirements for a XRootD-based data buffer between the single-phase
protoDUNE detector DAQ and central CERN computing are given.  Plans are made to
create a data source emulator equivalent to the  \pd DAQ, and also emulate several
elements of the CERN computing  environment in order to prototype the raw data management system for \pdd
This work shall be done at Brookhaven National Lab. A proposal for a system design to be built at CERN is presented.
Finally, a functional test of the Fermilab File Transfer System is described within the context of the \pd raw
data management.
\end{abstract}

 % \tableofcontents

\pagebreak


\section{Overview}
\subsection{Under Construction!}
We will fill in this document as we do the work.  Right now it is just
an outline.

\subsection{the protoDUNE program}
The description of the protoDUNE experimental program can be found in (provide quotes).

\section{Requirements and Assumptions}

This section will give:
\begin{itemize}
\item data rates based on the data taking ``scenarios'' spread sheet
\item CERN requirements (eg, 3 days buffer)
\item assumptions about various bandwidths (eg network and disk)
\item ...
\end{itemize}


\section{DAQ and Environment Model}

We will develop a simple, functionally equivalent emulation for the SP
DAQ data source and implement a model of the network environment from
DAQ to EOS.

\section{Results}

We will look at:

\begin{itemize}
\item scaling as a function of number of concurrent writes 
\item forced different bottlenecks at NIC and disk.
\item effect of simultaneous reads/writes to same box, bus, disk. 
\item ...
\end{itemize}

This test will be done at BNL, probably on existing RACF LBNE nodes.
For it we have identified 3 interactive nodes and 7 pure-batch nodes
with an existing round-robin XRootD installation.

\section{Design}

Based on results this section will provide 
\begin{itemize}
\item design for hardware (number of nodes, disk speeds, RAM, etc)
\item configuration
\item any limitations (eg, on number of simultaneous reads+writes)
\end{itemize}

\section{FTS Testing}

There are three parts to this testing.

\begin{enumerate}
\item Buffer farm $\to$ ``EOS''
\item ``EOS'' $\to$ FNAL
\item FNAL $\to$ ``OSG''
\end{enumerate}

The quotes are used as ``EOS'' will be emulated with a simple storage
node at BNL and ``OSG'' will simply again be nodes at BNL.  The goal
is a functional tests of all these types of transfers if not a
performance one.

More information about FTS is in the document ``Design of the Data
Management System for the protoDUNE Experiment'').

\section{References}
\bibliography{citedb}


\end{document}

%%% Local Variables:
%%% mode: latex
%%% TeX-master: t
%%% End:
